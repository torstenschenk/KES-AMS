\section{Gstreamer} \label{RefGstreamer}

%%%%%%%%%%%%%%%%%%%%%%%%%%%%%%%%%%%%%%%%%%%%%%%%%%%%%%%%%%%%%%%%%%%%%%%%%%
\subsection{Installation}
Mittels Paketmanager APT (Debian/Ubuntu), yum (Fedora/Centos) or homebrew (Mac) 
fehlende Pakete installieren, am Beispiel Ubuntu (Rapian Debian):
\begin{verbatim}
  sudo apt update 
  sudo apt install libgstreamer1.0-0 gstreamer1.0-plugins-base \
    gstreamer1.0-plugins-good gstreamer1.0-plugins-bad \
    gstreamer1.0-plugins-ugly gstreamer1.0-libav gstreamer1.0-doc \
    gstreamer1.0-tools
  sudo apt install libgstreamer1.0-dev
\end{verbatim}

%%%%%%%%%%%%%%%%%%%%%%%%%%%%%%%%%%%%%%%%%%%%%%%%%%%%%%%%%%%%%%%%%%%%%%%%%%
\textbf{I/O Elemente}
Ein Gstreamer Befehl ist aus einer Kette von Plugins aufgebaut. Die Verkettung der Befehle startet mit ein oder mehreren \textbf{src} Plugins und endet mit einem oder mehreren \textbf{sink} Plugins. Es folgt eine List mit src und sink Plugins.\\

\textbf{Beschreibung der verwendeten Eingangs oder SRC-Elemente}
\begin{itemize}
\item v4l2src – stream from a camera device on a linux system, e.g. device=/dev/video0;
\item audiotestsrc – used to do test streams with audio;
\item videotestsrc – used to do test streams with video, you may specify a pattern=<num>;
\item fakesrc – another option for testing by feeding in an empty stream;
\item filesrc – stream from a file, specifiy location=<filepath>;
\item ximagesrc – capture screen.
\end{itemize}

\textbf{Ausgabe oder SINK Elemente}
\begin{itemize}
\item filesink – save stream to a file, specify location=<filepath>;
\item  autoaudiosink – play audio on an automatically detected device;
\item  autovideosink – play video on an automatically detected display utility and device;
\item  fakesink – do not play stream, just finish;
\item  udpsink – stream result over UDP, specify host=<IP of the target server> and port=<number>;
\item rtmpsink – stream result over RTMP, specify host=<IP of the target server> and port=<number>.
\end{itemize}

\textbf{Kodierungselemente}\\
Die gstreamer Bibliothek umfasst Elemente um Streams zu komprimieren. Sie sind mit dem Kürzel 'enc' versehen, z.B. vorbis\textbf{enc}. Die Elemente zum Dekodieren eines Datenstroms verwenden dann das Kürzel 'dec' um die Daten zu dekomprimieren, z.B. vorbis\textbf{dec}.\\

\textbf{Audio}
\begin{itemize}
\item mp3 – lamemp3enc, avenc\_mp3 | mad, mpg123audiodec, avdec\_mp3; 
\item aac – voaccenc, faac, avenc\_aac | faad, aacparse, avdec\_aac;
\item vorbis – vorbisenc | vorbisdec, vorbisparse;
\item opus – opusenc, avenc\_opus | opusdec, avdec\_opus.
\end{itemize}

\textbf{Video}
\begin{itemize}
\item h.264 – x264enc, avh264\_enc |  h264parse, mpeg4videoparse, avdec\_h264;
\item mpeg2 -mpeg2enc, avenc\_mpeg2video | mpeg2dec, avdec\_mpeg2video;
\item jpeg2000 – no inter-frame coding, low latency; avenc\_jpeg2000 | avdec\_jpeg2000;
\item vp8 – vp8enc, avenc\_vp8 | vp8dec, avdec\_vp8;
\item vp9 – vp9enc, avenc\_vp9 | vp9dec, avdec\_vp9;
\item theora -theoraenc | theoradec, theoraparse.
\end{itemize}

%%%%%%%%%%%%%%%%%%%%%%%%%%%%%%%%%%%%%%%%%%%%%%%%%%%%%%%%%%%%%%%%%%%%%%%%%%

\textbf{Weitere Bestandteile der Pipeline}
\begin{itemize}
\item \textbf{Payers/Depayers} – Diese Elemente bereiten die (payload) Datapakete vor und holen die Daten nach dem Transport durchs Netzwerk wieder heraus. E.g.: rptvp8pay/rtpvp8depay, \\ rtph264/rtph264depay
\item \textbf{Konvertierer} – Diese Elemente können verwendet werden, um Manipulationen, wie z.B. Rotation, Farbraumtransformation, Skalierung auszuführen.
E.g.: audioconvert, audioresample, videoconvert, videoscale.
\end{itemize}
\color{black}
%%%%%%%%%%%%%%%%%%%%%%%%%%%%%%%%%%%%%%%%%%%%%%%%%%%%%%%%%%%
\subsection{Gstreamer Tests für Video \& Audio}

\textbf{Synthetische Quellen}

Video Test Source zum Display
\begin{verbatim}
Testbild:
  gst-launch-1.0 videotestsrc pattern=1 ! videoconvert ! autovideosink
Webcam oder intergrierte Kamera: 
  gst-launch-1.0 autovideosrc device=/dev/video0 ! autovideosink
\end{verbatim}

Audio Test Source zum Lautsprecher
\begin{verbatim}
  gst-launch-1.0 audiotestsrc ! audioconvert ! autoaudiosink
\end{verbatim}

Audio Test Quelle zu Fake Sink, aber trotzdem vollständige Pipeline
\begin{verbatim}
  gst-launch-1.0 audiotestsrc ! audioconvert ! fakesink
\end{verbatim}

Video Test Broadcast über TCP/HTTP
\begin{verbatim}
Sender
  gst-launch-1.0 videotestsrc horizontal-speed=5  ! vp8enc ! gdppay ! \
    tcpserversink host=127.0.0.1 port=5200
Empfänger
  gst-launch-1.0 -v tcpclientsrc port=5200 ! gdpdepay ! vp8dec ! \
    videoconvert ! autovideosink
\end{verbatim}

\textbf{Video Broadcast über RTP (via UDP) von der Webcam}
\begin{verbatim}
Sender
  gst-launch-1.0 v4l2src ! videoconvert ! video/x-raw, width=640,height=480 ! \
    omxh264enc ! rtph264pay pt=96 config-interval=1 ! \
    udpsink host=192.168.2.106 port =8554
Empfänger
  gst-launch-1.0 udpsrc port=8554 caps="application/x-rtp,media=video,\
    clockrate=90000,payload=96,encoding-name=H264" ! \
    rtph264depay ! avdec_h264 ! autovideosink
\end{verbatim}

\textbf{Video mit synthetischem Audio}
\begin{verbatim}
Sender:
  gst-launch-1.0 -v audiotestsrc ! audioconvert ! \
    audio/x-raw,channels=1,depth=16,width=16,rate=44100 ! rtpL16pay pt=97 ! \
    udpsink host=192.168.178.29 port=5001 v4l2src ! videoconvert ! \
    video/x-raw, width=640,height=480 ! omxh264enc ! \
    rtph264pay pt=96 config-interval=1 ! \
    udpsink host=192.168.178.29 port=5000
Empfänger
  gst-launch-1.0 udpsrc port=5001 ! application/x-rtp, clock-rate=44100, payload=97 ! 
    rtpL16depay ! audioconvert ! alsasink sync=false udpsrc port=5000 
    caps="application/x-rtp,media=video,payload=96,encoding-name=H264" ! rtph264depay !     
    avdec_h264 ! autovideosink
\end{verbatim}

\subsection{Kamera \& Mikrofon Streaming}
Erster Kamera Test:
\begin{verbatim}
  gst-launch-1.0 v4l2src ! xvimagesink
  gst-launch-1.0 v4l2src ! jpegdec ! xvimagesink
\end{verbatim}

\textbf{Video und Audio einer Webcam über RTP (in UDP Paketen)}
\begin{verbatim}
Sender
  gst-launch-1.0 -v alsasrc device=plughw:CARD=StudioTM,DEV=0 ! \
    audioconvert ! audio/x-raw,channels=1,depth=16,width=16,rate=44100 ! \
    rtpL16pay pt=97 ! udpsink host=192.168.178.29 port=5001 v4l2src ! \
    videoconvert ! video/x-raw, width=640,height=480 ! omxh264enc ! \
    rtph264pay pt=96 config-interval=1 ! udpsink host=192.168.178.29 port=5000
Empfänger
  gst-launch-1.0 udpsrc port=5001 ! application/x-rtp, clock-rate=44100,\
    payload=97 ! rtpL16depay ! audioconvert ! alsasink sync=false udpsrc \
    port=5000 caps="application/x-rtp,media=video,payload=96,encoding-name=H264" ! \
    rtph264depay ! avdec_h264 ! autovideosink
\end{verbatim}

\textbf{Raspi WebCam C920 Video/Audio an Webserver via RTP (UDP)}
\begin{verbatim}
Sender
  gst-launch-1.0 -v alsasrc device=plughw:1,0 ! audioconvert \
    ! audio/x-raw,channels=1,depth=16,width=16,rate=44100 \
    ! rtpL16pay pt=97 ! udpsink host=85.214.211.169 port=5001 v4l2src \
    ! videoconvert ! video/x-raw, width=640,height=480 ! omxh264enc \
    ! rtph264pay pt=96 config-interval=1 ! udpsink host=85.214.211.169 port=5000
\end{verbatim}

%%%%%%%%%%%%%%%%%%%%%%%%%%%%%%%%%%%%%%%%%%%%%%%%%%%%%%%%%%%%%%%%%%%%%%%%%%
\section{Janus-Gateway} \label{RefJanus}
Janus ist ein Server, der WebRTC Medien Kommunikation mit einem Browser unterstützt und json Nachrichten versendet. Zusätzlich können auch RTP/RTCP und Nachrichten zwischen Browsern und serverseitigen Programmen ausgetauscht werden. Plugins ermöglichen z.B. auch den Empfang von Streams im RTP/UDP Format versendet von Gstreamer. 

\subsection{Installation von Janus}
\begin{verbatim}
sudo apt install slibmicrohttpd-dev libjansson-dev libnice-dev \
	libssl-dev libsrtp-dev libsofia-sip-ua-dev libglib2.0-dev \
	libopus-dev libogg-dev libcurl4-openssl-dev liblua5.3-dev \
	pkg-config gengetopt libtool automake

git clone https://github.com/meetecho/janus-gateway.git
cd janus-gateway
sh autogen.sh

./configure --prefix=/opt/janus

sudo make 
sudo make install 
make configs

wenn ein libsrtp Fehler angezeigt wird libsrtp aus dem Quelen bauen:

wget https://github.com/cisco/libsrtp/archive/v2.0.0.tar.gz
tar xfv v2.0.0.tar.gz
cd libsrtp-2.0.0
./configure --prefix=/usr --enable-openssl
make shared_library && sudo make install
\end{verbatim}

%%%%%%%%%%%%%%%%%%%%%%%%%%%%%%%%%%%%%%%%%%%%%%%%%%%%%%%%%%%%%%%%%%%%%%%%%%%%%
\subsection{Konfiguration}
Zum Streaming via gstreamer RTP muss noch die cfg Datei angepasst werden.\\
Alle cfg Dateien liegen in: /opt/janus/etc/janus\\

In Datei: janus.transport.http.cfg
\begin{verbatim}
[general]
http = yes
ip = ...server-addr-ip-hostname
[admin]
admin_http = yes
\end{verbatim}

janus.plugin.streaming.cfg
\begin{verbatim}
[gst-rpwc]
type = rtp 
id = 1 
description = RPWC H264 test streaming 
audio = yes 
audioport = 8005 
audiopt = 10 
audiortpmap = opus/48000/2 
video = yes 
videoport = 8004 
videopt = 96 
videortpmap = H264/90000 
videofmtp = profile-level-id=42e028\;packetization-mode=1 
\end{verbatim}

%%%%%%%%%%%%%%%%%%%%%%%%%%%%%%%%%%%%%%%%%%%%%%%%%%%%%%%%%%%%%%%%%%%%%%%%%%%%%
\subsection{nginx Webserver} \label{Refnginx}
Installation von nginx Webserver zum Webseiten Testen in lokalem Netzwerk\\
Frei nach: https://www.digitalocean.com/community/tutorials/how-to-install-nginx-on-ubuntu-16-04
\begin{verbatim}
  sudo apt-get update
  sudo apt-get install nginx
  sudo ufw app list
\end{verbatim}

You should get a listing of the application profiles:\\
Terminal Ausgabe
\begin{verbatim}
Available applications:
  Nginx Full
  Nginx HTTP
  Nginx HTTPS
  OpenSSH
\end{verbatim}

Es ist meistens nötig Traffic auf dem Port 80 zu erlauben.\\
Man kann es konfigurieren mittels:
\begin{verbatim}
  sudo ufw allow 'Nginx HTTP'
  sudo ufw status
  http://server_domain_or_IP
\end{verbatim}

nginx Beispiele von janus nach /usr/share/nginx/html/demos kopieren
\begin{verbatim}
  sudo cp -r /opt/janus/share/janus/demos /usr/share/nginx/html/
\end{verbatim}
Im Browser unter http://ip-server/demos
wird trotzdem nur die Platzhalter Webseite angezeigt.

Um auf die kopierten Janus Webseiten zuzugreifen, muss noch etwas in der nginx Konfiguration geändert werden: /etc/nginx/sites-available
\begin{verbatim}
 #root /var/www/html;
 root /usr/share/nginx/html;
\end{verbatim}
Nun können die  Janus Beispiel html Dateien geladen werden.\\
http://192.168.178.25/demos/\\

Der Admin Monitor ist nur nach Eingabe eine Passwort verfügbar...\\
Entweder das Passwort aus /opt/janus/etc/janus/janus.cfg verwenden \\
(admin\_secret = janusoverlord) oder einfach wie folgt anpassen:
\begin{verbatim}
/usr/share/nginx/html/demos/admin.js
line 14
var secret = "janusoverlord";
\end{verbatim}
Danach ist das Passwort schon voreingestellt und wird nicht mehr abgefragt.\\
Nun kann der Admin Bereich im Browser geöffnet werden:
\begin{verbatim}
http://ip-server/demos
\end{verbatim}

%%%%%%%%%%%%%%%%%%%%%%%%%%%%%%%%%%%%%%%%%%%%%%%%%%%%%%%%%%%%%%%%%%%%%%%%%%%%%
\subsection{gstreamer an janus} \label{RefGstrToJanus}
Gstreamer Kommandos, um vom Raspberry V/A an das Janus-Gateway zu streamen\\
Zwei verschiedene Webcams wurden gestetet. \\
(1) Webcam HD Pro: CARD=C920,DEV=0 \\
(2) Microsoft® LifeCam Studio(TM): Webcam CARD=StudioTM,DEV=0 \\
Webcam (2) unterstützt h264parse nicht. Die Pipeline wurde alterntiv mit 
omxh264enc ergänzt. 
\begin{verbatim}
Webcam HD Pro: CARD=C920,DEV=0
  gst-launch-1.0 -v v4l2src ! h264parse ! rtph264pay config-interval=1 pt=96 ! \
    udpsink host=192.168.178.25 port=8004 alsasrc device=plughw:1,0 ! \ 
    audioconvert ! audioresample ! opusenc ! rtpopuspay ! udpsink \ 
    host=192.168.178.25 port=8005
 
Microsoft® LifeCam Studio(TM): Webcam CARD=StudioTM,DEV=0
Mit Bildskalierung, aber langsames Streamig (häufige Aussetzer)
  gst-launch-1.0 v4l2src ! videoconvert ! video/x-raw, width=640,height=480 !\
    omxh264enc ! rtph264pay pt=96 config-interval=1 ! \ 
    udpsink host=192.168.178.29 port=8004 alsasrc device=plughw:1,0 ! \
    audioconvert ! audioresample ! opusenc ! rtpopuspay ! udpsink \
    host=192.168.178.29 port=8005 
   
Microsoft® LifeCam Studio(TM): Webcam CARD=StudioTM,DEV=0
Ohne Bildskalierung, trotzdem schnelleres Streaming
  gst-launch-1.0 v4l2src ! omxh264enc ! rtph264pay pt=96 config-interval=1 ! \ 
    udpsink host=192.168.178.25 port=8006 alsasrc device=plughw:1,0 ! \
    audioconvert ! audioresample ! opusenc ! rtpopuspay ! udpsink \
    host=192.168.178.25 port=8007 
\end{verbatim}

%%%%%%%%%%%%%%%%%%%%%%%%%%%%%%%%%%%%%%%%%%%%%%%%%%%%%%%%%%%%%%%%%%%%%%%%%%