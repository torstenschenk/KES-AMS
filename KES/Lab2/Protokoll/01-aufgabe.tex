\section{Vorwort}
Bei der Recherche zur Bearbeitung der Übungen wurden viele englischsprachige Webseiten zu rate gezogen. Generell kann man sagen, dass englische Fachbegriffe sich im Bereich FPGA und embedded Design etabliert haben, so dass eine Übersetzung eher verwirren als helfen würde. Daher haben wir uns entschieden, die \textbf{englischen} Bezeichner und Beschreibungen beizubehalten.\\
Um Codeabschnitte besser von Beschreibungen besser unterscheiden zu können, wurde eine eigene Schriftart verwendet:
\begin{verbatim}
  Kommandozeilen Eingaben und Codesnippets werden wie HIER dargestellt.
\end{verbatim}

\section{Aufgabe 1} \label{ex1}
In der Laborübung wurde das ZedBoard Zynq-7000 eingesetzt. Es umfasst als \textbf{PL} den Artix-7 FPGA mit 85K Logic Cells (Device Z-7020, Part: XC7Z020) und als \textbf{PS} den Dual-core ARM Cortex-A9 MPCore™ mit 866 MHz.\\

\textbf{Isola}\\
Auf der TODO
